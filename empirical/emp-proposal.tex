\documentclass[american, abstract=on]{scrartcl}

    \newcommand{\lang}{en}

    \usepackage{babel}
    \usepackage[utf8]{inputenc}

    \usepackage{csquotes}

    \usepackage{amsmath, amssymb, mathtools, bbm, bm}
    \usepackage{xcolor}
    \usepackage{xcolor-solarized}
    \usepackage{graphicx}
    \usepackage{wrapfig}
    \usepackage{relsize}
    \usepackage{makecell}
    \usepackage{booktabs}
    \usepackage[font=footnotesize,labelfont=bf]{caption}
    \usepackage{subcaption}
    \usepackage{float}
    \usepackage{multirow} 

    % Refs
    \usepackage{hyperref}
    \usepackage{cleveref}
    \hypersetup{
        colorlinks = true, 
        urlcolor = blue,
        linkcolor = blue, 
        citecolor = blue 
      }      

    \usepackage{subfiles} % Load last

    % Paths

    % Formatting
    \setlength{\parindent}{0em}
    \setlength{\parskip}{0.5em}
    \setlength{\fboxsep}{1em}
    \newcommand\headercell[1]{\smash[b]{\begin{tabular}[t]{@{}c@{}} #1 \end{tabular}}}

    % Graphs

    % Math commands

    \newcommand{\diff}{\text{d}}
    \renewcommand{\Re}{\mathbb{R}}
    \newcommand{\C}{\mathcal{C}}
    \newcommand{\F}{\mathcal{F}}
    \newcommand{\X}{\mathcal{X}}
    \newcommand{\G}{\mathcal{G}}
    \newcommand{\I}{\mathcal{I}}
    \newcommand{\N}{\mathcal{N}}
    \newcommand{\PF}{\mathcal{P} \F}

    \renewcommand{\P}{\mathbb{P}}
    \newcommand{\E}{\mathbb{E}}

    \newcommand{\uI}[2][s]{\int^1_0 #2 \ \text{d} #1}
    \newcommand{\uH}[2][s]{\int^\frac{1}{2}_0 #2 \ \text{d} #1}
    \newcommand{\uF}[2][s]{\int^1_\frac{1}{2} #2 \ \text{d} #1}
    \newcommand{\norm}[1]{\left\lVert#1\right\rVert}
    \newcommand{\abs}[1]{\left\lvert#1\right\rvert}

    % Bibliography

    \usepackage[bibencoding=utf8, style=apa, backend=biber]{biblatex}
    \addbibresource{empirical-production-networks.bib}

    \newcommand{\citein}[1]{\citeauthor{#1} (\citeyear{#1})}

    \newcommand\notes[1]{\textcolor{teal}{\textbf{#1}}}
    \newcommand\red[1]{\textcolor{red}{#1}}

    % Make title page

    \author{Andrea Titton}
    \title{Suppliers' Risk, Multi-Sourcing, And Production Networks Fragility}
    \subtitle{Proposal}
    
\begin{document}

\maketitle

\section{Introduction}

The role of firms decisions in generating aggregate production network risk has long been recognized. In picking its suppliers, each firm does not account for the risk it imposes on downstream producers. This can lead to under-sourced and fragile production networks (\cite{steve_banker_if_2020}). Furthermore, firms are often unaware of how correlated their potential suppliers' risks are. In light of this, I plan to use firm level data from the Netherlands to study the role of suppliers' uncertainty in firms decisions and aggregate fragilities.


\section{Methodological Literature Review}

In recent years, there has been a surge in interest on the effect of firm level idiosyncrasies on aggregate fluctuations of production networks. The standard approach in the literature is to construct a model consisting of firms that are heterogeneous in production technology and make supplier decisions endogenously. These models are then calibrated by matching industry aggregates, such as degree distributions. Finally, the calibrated model is employed to simulate counterfactuals.

An early example of an empirical analysis of the link between microeconomic and macroeconomic shocks is \citein{carvalho_supply_2016}. The authors use the commodity-by-commodity US Requirements Tables to infer production functions between goods. Finally, they draw an edge between two industries if the supplier accounts for more than 5\% of the total inputs of a good. Finally, they calibrate the model parameters to match the distribution of the weighted out-degree observed in the dataset. Using this model, the authors simulate productivity shocks and their aggregate consequences. They find that industries with larger eigenvector centralities contribute more to aggregate fluctuations.\footnote{There is also an important game theoretical literature where endogenous discontinuities are derived in simplified production games, for example \citein{elliott_supply_2022} and \citein{amelkin_strategic_2020}}

A similar, but more granular, approach has been employed by \citein{mathieu_taschereau-dumouchel_cascades_2020}\footnote{The workhorse model has been developed by \citein{acemoglu_endogenous_2020}}. The authors use the FactSet Revere US firm-level production network data to calibrate an endogenous production network model with uncertainty around suppliers productivity. The authors show that, allowing for reallocations tames the size of aggregate shocks and fluctuations in production networks. 

An example of a model agnostic approach is that of \citein{barrot_input_2016}. The authors use natural disasters to identify the role of industries in aggregate fluctuations. They find that when a natural disaster hits a supplier, firms experience an average drop by 2 to 3 percentage points in sales growth, even if the supplier represents a small share of firms' total intermediate inputs. This effect is proportional to input specificity.\footnote{Other important model agnostic analysis has been conducted by \citein{luttmer_selection_2007}, \citein{atalay_network_2011}, and \citein{mackay_how_2020}}

\section{Proposal}

Guided by a model of endogenous production networks (e.g. \cite{kopytov_endogenous_2021,elliott_supply_2022}), I plan to leverage supply chain disruptions experienced by firms, to identify the degree of suppliers' diversification among firms in the Netherlands and to what extent this is driven by opacity of supply chain risk. 

This can be done by using firm level input output data. If the supplier risk is not properly diversified, the correlation among firms' suppliers' risk within an industry would yield an overdispersed distribution of output following the shock, vis-à-vis the case of perfect diversification. Such an overdispersion should compound as the shock propagates through industries. With the CBS data it would be possible to estimate the overdispersion of output and its propagation among Dutch firms following a supply chain disruption and develop a counterfactual ``perfect diversification''. Estimating the degree of diversification of the production network as a whole would help policymakers better targeted interventions that could endogenously allow for more resilience without compromising economic efficiency.

\section{Potential Data Limitations}

\begin{enumerate}
  \item A first concern that arises is whether data imputation might bias the estimation of the overdispersion of output distribution.
  \item Another potential issue arises from not having sufficiently frequent data. I believe yearly data is sufficient (year before the shock and year of the shock).
\end{enumerate}

% --- Bibliography
\newpage
\printbibliography




\end{document}