\documentclass[../../main.tex]{subfiles}
\begin{document}

\section{Setup}

\subsection{Notation and Distributions}

To study firms' decision in an opaque production network, I assume firms reason probabilistically about what is, at the core, a combinatorial problem. To talk about such a model, it is useful to introduce some notation and distributions that play a central role in describing the reasoning and decisions of firms. Consistently with the combinatorial and graph theory literature (\cite{harris_combinatorics_2008}), for a real number $x$ and a positive integer $n$, I  denote the rising factorial as

\begin{equation}
  (x)^{(n)} \coloneqq \underbrace{x(x + 1)(x + 2)\ldots(x + n - 1)}_{n \text{ terms}}.
\end{equation}

Abusing notation, I  often extend this function to real exponents $s$ by using the $\Gamma$ function, such that,

\begin{equation}
  (x)^{(n)} \coloneqq \frac{\Gamma(1 + x)}{\Gamma(1 + x - s)}.
\end{equation} It is not hard to see that the two definitions are equivalent over the integers.

\notes{TODO: Introduce beta and beta-binomial distribution}

\subsection{Model}

In this section I introduce the structure of the model and the assumptions that allow it to describe firms sourcing decisions when not all information about the supply chain realisation is observable. Consider a vertical economy producing $K + 1$ goods (as displayed in Figure \ref{fig:vertical-economy-diagram}), indexed by $k \in [K] \coloneqq \{ 0, 1, 2, \ldots K \}$. Each firm produces a single good and each good is produced by $m$ firms. Good $k$ requires only good $k - 1$ as input, hence I  refer to them as good or layers interchangibly. In this paper I do not focus on complex goods (as done in \cite{elliott_supply_2022}) or on the role of cycles in the production network (\cite{acemoglu_network_2012,baqaee_macroeconomic_2019}), but restrict my attention to vertical economies producing simple goods to isolate the interplay of suppliers' correlation and unobservability of the production network. Each firm picks a set of suppliers in the previous layer to source from. Establishing a relation with a supplier has a fixed cost $\kappa$. If no supplier is able to deliver the input good, then the firm is not ``functional'' and hence not able to deliver downstream. I assume that being functional yields an exogenous payoff $\pi$. This assumption can be relaxed by introducing market structure to endogenise $\pi$, yet this does not change the main model mechanics. Finally, I assume that firms know the structure of the economy but do not observe the realised supplier relationship in upstream layers. The only source of risk in the model is an idiosyncratic probability $\mu_0$ that basal firms in the zeroth layer are not able to carry on production.
\begin{figure}[H]
  \centering
  \begin{subfigure}{.5\textwidth}
    \centering
    \inputTikZ{0.6}{../diagrams/model-presentation.tikz} 
    \caption{$K$-layers vertical economy}
    \label{fig:vertical-economy-diagram}
  \end{subfigure}%
  \begin{subfigure}{.5\textwidth}
    \centering
    \inputTikZ{0.6}{../diagrams/model-realisation.tikz} 
    \caption{One instance of a supply chain realisation}
    \label{fig:vertical-economy-diagram:suppliers}
  \end{subfigure}
  \caption{Example of a vertical economy}
  \label{fig:gf}
\end{figure}


A firm is identified by a tuple $(k, i)$, where $i \in [m]$ is an index and $k \in [K]$ is its good or layer. Letting $\mathcal{F}_k$ be the set of functioning firms in layer $k$ and $\mathcal{S}_{k, i}$ the set of suppliers of firm $(k, i)$, the probability that such firm is functioning is

\begin{equation} \label{eq:functional_probability}
  p_{k, i} \coloneqq \P\big( i \in \mathcal{F}_k \big) = 1 - \P\big( \mathcal{F}_{k - 1} \cap \mathcal{S}_{k, i} = \emptyset \big),
\end{equation}

since the event $\left\{ \mathcal{F}_{k - 1} \cap \mathcal{S}_{k, i} = \emptyset \right\}$ implies that no suppliers of firm $(k, i)$ are functional.

\end{document}