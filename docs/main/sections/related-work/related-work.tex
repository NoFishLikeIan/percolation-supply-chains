\documentclass[../../main.tex]{subfiles}
\begin{document}

\section{Related Work}

In this section, I place the framework within the existing literature and highlight its major contributions. In recent years, supply chain disruptions have renewed interest in the role firm level choices have in generating aggregate disruptions, particularly via production networks formations. This question can be traced as far back as \citein{jovanovic_micro_1987}, \citein{hulten_growth_1978}, and \citein{horvath_cyclicality_1997}\footnote{Other important contributions are for example \citein{leontief_quantitative_1936} and \citein{kremer_o-ring_1993}}. These seminal contribution demonstrates how the production (network) structure plays a crucial role in propagating and aggregating firm idiosyncratic shocks. More recently, the mechanism behind the propagation of firm level shocks has been further investigated by \citein{acemoglu_network_2012}, \citein{baqaee_cascading_2018}, and \citein{baqaee_macroeconomic_2019}. Importantly the authors show how second or higher order effects are crucial in assessing how a shock will propagate from firms to industries and, further, to the whole economy. 

The literature presented above does not deal explicitly with the formation of the production network, which is usually imposed exogenously. Nevertheless, modelling the drivers behind firms' sourcing decisions is crucial in studying the propagation of firm level shocks. First, firms make sourcing decision based on the perceived risk of their suppliers, such that, not all possible production network configuration do arise. Second, firms react to changing economic conditions, such as changes in technologies or uncertainty, such that, the production network configuration changes in response to a shcok. These two factors imply that production networks are endogenous to the shocks and, hence, shock propagation cannot be studied conditional on a particular network realisation. Important work has been done on endogenous production network to understand their role in enabling long term aggregate growth, most notably by \citein{acemoglu_endogenous_2020}. 

This paper focuses more on the role of endogenous production networks in explaining their fragility, as studied, among others, by \citein{erol_network_2014}, \citein{elliott_supply_2022}, and \citein{kopytov_endogenous_2021}. In this paper, I study the role of imperfect informations of the supply chain network in determining the aggregate network fragility. The modelling approach is very close to that employed in \citein{erol_network_2014} and \citein{elliott_supply_2022}, in that the production function is highly stylised. 

\end{document}