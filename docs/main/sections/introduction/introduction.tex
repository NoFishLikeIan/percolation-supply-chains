\documentclass[../../main.tex]{subfiles}
\begin{document}

\section{Introduction}

\iffalse
\notes{
  \begin{enumerate}
    \item Research question
    \item Contribution to the literature
    \item Why do we care?
  \end{enumerate}
}
\fi

Firms operating in a production network often need to diversify sources to deal with supply chain risk. Nevertheless, the structure of the supply chain, beyond their immediate suppliers, is usually unobservable. In this paper, I construct a stylised production game to study the relationship between the opacity of the supply chain, firms' diversification decisions, and production network fragility. In the model, unobserved correlation among suppliers generates fragility via two channels. First, it directly introduces endogenous correlation in downstream firms' risk, which propagates through the production network. That increase the probability of cascading failures in which the entire production network is unable to produce. Second, it indirectly affects firms' decisions by reducing the marginal gain they can expect from adding a source of inputs. The latter channel leads to firms diversifying increasingly less, such that, arbitrary small shocks in basal risk can generate cascading failures downstream. 

Supply chain disruption have become a central concern for firms and policymakers \notes{add policy relevance} 

Consistently with existing literature (e.g. \cite{elliott_supply_2022,kopytov_endogenous_2021}), in equilibirum small idiosyncratic shocks can be massively amplified. The degree of amplification depends on the equilibirum behaviour of firms. This phenomenon holds true in vertical economies producing simple goods. The novel theoretical contribution of this paper is to extend the analysis of production network formation to an opaque environment in which firms aim to minimise risk accounting for suppliers' correlation. To do so, I develop a tractable analytical model that describes the propagation of idiosyncratic shocks through the supply chain. By describing the evolution of firms' failure to produce as a dynamical system over the layers of the supply chain, as opposed to time, the model is able to keep track of the entire risk distribution. Furthermore, I show that the model can be seen as an extension to large but finite supply chains of the one developed by \citein{elliott_supply_2022}.



\end{document}