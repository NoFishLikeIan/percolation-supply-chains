\documentclass[../../main.tex]{subfiles}
\begin{document}


\section{Social Planner}

The social planner problem consists in, on the one hand, minimising aggregate risk and, on the other, the number of edges. We can write aggregate expected welfare as a function of the sequence of suppliers' sets $\mathcal{S}_{k, i}$ by summing the individual firms' payoffs

\begin{equation} \label{eq:social-planner:complete-W}
  W(\mathcal{S}_{1, 0}, \mathcal{S}_{2, 0}, \ldots, \mathcal{S}_{K, m}) \coloneqq \sum^{(K, m)}_{(k, i) = (1, 0)} \Big(1 - \P\big( \mathcal{F}_{k - 1} \cap \mathcal{S}_{k, i} = \emptyset \big) \Big) \ \pi  - \kappa \abs{\mathcal{S}_{k, i} }.
\end{equation}

Maximising the function $W$, with respect to the sequence $\left\{ \mathcal{S}_{k, i}\right\}_{(k, i)}$, requires the social planner to dictate to each individual firm which suppliers to source from. This demands a lot of information, which are hidden from the firm, and of regulatory power. Hence, in addition to solving the planners' problem consisting of maximising equation (\ref{eq:social-planner:complete-W}), we will consider the problem in which the planner is only able to coherce the firm into picking a given number of suppliers and the firm is free to pick which suppliers to source from. We can write this constrained social planner problem in the ``language'' of our model as maximising

\begin{equation}
  W_c(s_1, s_2, \ldots, s_k) \coloneqq m \ \sum^{K}_{k = 1} \Big(1 - G_1(\mu_k, \rho_k, s_k)\Big) \pi - \kappa \  s_k.
\end{equation}

\subsection{Unconstrained Social Planner Problem} \label{sec:planner:unconstrained}

The risk faced by a firm depends only on how many firms in the first layer it is connected to, regardless of what path. Namely, if $n$ basal firms are involved in the firms' production, its risk is $1 - \mu^n$. Hence, to find the welfare maximising sequence of suppliers, we can first look for the most edge parsimonious way to achieve a given level of risk $\mu^n$, such that

\begin{equation}
  \P\big( \mathcal{F}_{k - 1} \cap \mathcal{S}_{k, i} = \emptyset \big) \equiv  \mu^n \text{ for all } (k, i),
\end{equation}

and then find the optimum $n$. The most edge parsimonious graph that achieves $\mu^n$ risk is the $\textit{min-max}n$ graph.

\begin{definition}
  Let $\textit{min-max}(n)$ be the network where all firms in layer $1$ have $n$ suppliers and thereafter, each firm, is connect to only one supplier. That is
  
  \begin{equation}
    \mathcal{S}_{k, i} = \begin{cases}
      \{(0, 0), (0, 1), (0, 2), \ldots, (0, n - 1)\} &\text{ if } k = 1,\\
      \{(k - 1, i)\} &\text{otherwise.}
    \end{cases}
  \end{equation}
\end{definition}

\begin{lemma}
  $\textit{min-max}(n)$ is the network with fewest edges that achieves $\mu^n$ risk.
\end{lemma}

\begin{proof}

To see this, consider Figure (\ref{fig:planner-n-target}). If the planner wants to achieve risk $\mu^n$ in layer $k$, any branching (right), vis-à-vis the min-max network (left), requires at a list one more edge to close the branching, hence it has as strictly more edges than the min-max network. Furthermore, note that if a firm in layer $k$ is connected, the marginal benefit of connecting it $\pi (1 - \mu^n)$ must have been bigger than marginal costs $\kappa$, then it must be profitable to connect a firm in layer $k + 1$, since both marginal benefits and costant are the same, or equivalently, $\mathcal{S}_{k, i} = \{(k-1, i)\}$ implies that $\mathcal{S}_{k + 1, i} = \{(k, i)\}$. This allows us to exclude ``truncated'' $\textit{min-max}(n)$ networks.

\begin{figure}[H]
  \centering
  \begin{subfigure}{.5\textwidth}
    \centering
    \inputTikZ{0.5}{../diagrams/min-max-n.tikz} 
  \end{subfigure}%
  \begin{subfigure}{.5\textwidth}
    \centering
    \inputTikZ{0.5}{../diagrams/not-min-max-n.tikz} 
  \end{subfigure}%
  \caption{On the left the min-max$(n)$ network and on the right a deviation that achieves $\mu^n$.}
  \label{fig:planner-n-target}
\end{figure}

\end{proof}


Now that we know that min-max$(n)$ is the optimal graph to achieve risk $\mu^n$, we can look for the optimum $\mu^n$. Conditional on using the $\textit{min-max}(n)$ network, we can the welfare function (\ref{eq:social-planner:complete-W}) as a function of $n$,

\begin{equation}
  W(\textit{min-max}(n)) = m \pi (1 - \mu^n) - m \kappa,
\end{equation}

which is maximised for 

\begin{equation}
  n = \floor{ \frac{\log(\pi / \kappa) - \log(1 - \mu)}{\log(\mu)} }.
\end{equation}

\notes{Add implications}

\subsection{Constrained Social Planner Problem} \label{sec:planner:constrained}

The constrained social planner problem does not have an analytical solution but it has a Bellman formulation. Particularly, the value function in layer $k$ can be written as

\begin{equation}
  V_k(\mu_k, \rho_k) = \max_{s_k} \left\{ \pi - G_1(\mu_k, \rho_k, s_k) \ \pi - \kappa \  s_k + V_{k + 1}(G(\mu_k, \rho_k, s_k)) \right\}.
\end{equation}

\notes{Fix Belman formulation}

\end{document}