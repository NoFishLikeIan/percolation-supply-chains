\documentclass[../../main.tex]{subfiles}
\begin{document}


\section{Social Planner}

The social planner problem consists in, on the one hand, minimising aggregate risk and, on the other, the number of edges. The risk faced by a firm depends only on how many firms in the first layer it is connected to, regardless of what path. Namely, if $n$ basal firms are involved in the firms' production, its risk is $1 - \mu^n$. Hence, a natural question is, if the social planner seeks to connect a firm to $n$ basal firms, what is the most edge parsimonious way to do so? 

\begin{definition}
  Let $\textit{min-max}(n)$ be the network where all firms in layer $1$ have $n$ suppliers and thereafter, each firm, is connect to only one supplier. 
\end{definition}

\begin{lemma}
  $\textit{min-max}(n)$ is the network with fewest edges that achieves $\mu^n$ risk.
\end{lemma}

\begin{proof}

To see this, consider Figure (\ref{fig:planner-n-target}). If the planner wants to achieve risk $\mu^n$ in layer $k$, any branching (right), vis-à-vis the min-max network (left), requires at a list one more edge to close the branching, hence it has as strictly more edges than the min-max network.

\end{proof}

\begin{figure}[H]
  \centering
  \begin{subfigure}{.5\textwidth}
    \centering
    \inputTikZ{0.5}{../diagrams/min-max-n.tikz} 
  \end{subfigure}%
  \begin{subfigure}{.5\textwidth}
    \centering
    \inputTikZ{0.5}{../diagrams/not-min-max-n.tikz} 
  \end{subfigure}%
  \caption{On the left the min-max$(n)$ network and on the right a deviation that achieves $\mu^n$.}
  \label{fig:planner-n-target}
\end{figure}

Furthermore, note that if a firm in layer $k$ is connected, the marginal benefit $\pi (1 - \mu^n)$ must be bigger than marginal costs $\kappa$, then it must be profitable to connect a firm in layer $k + 1$, since the marginal benefit would be the same. Now that we know that min-max$(n)$ is the optimal graph to achieve $\mu^n$, we can look for the optimum $\mu^n$. Particularly, the planner will pick the largest $n$ such that

\begin{equation}
  n < \frac{\log(\pi / \kappa) - \log(1 - \mu)}{\log(\mu)}.
\end{equation}

\end{document}