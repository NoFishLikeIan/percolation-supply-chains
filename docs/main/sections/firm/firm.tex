\documentclass[../../main.tex]{subfiles}
\begin{document}


\section{Firm Problem}

In this section I lay down the problem of the firms. Furthermore, I leverage its symmetries to derive an analytical expression that describes the risk propagation in the supply chain when firms' sourcing decisions are endogenous. ,The problem of the firm $(k, i)$ is to maximise the expected payoff, which, using equation (\ref{eq:functional_probability}), we can write as

\begin{equation}
  \Big(1 - \P\big( \mathcal{F}_{k - 1} \cap \mathcal{S}_{k, i} = \emptyset \big) \Big) \ \pi  - \kappa \abs{\mathcal{S}_{k, i} },
\end{equation}

by picking a set of suppliers from those producing the previous good $\mathcal{S}_{k, i} \subseteq \{k - 1\} \times [m]$


To model supply chain uncertainty, we assume that a firm cannot observe, before making its supplier decision, the supplier decisions of the firms producing the necessary input good. Nevertheless, firms have knowledge of the position they occupy within the supply chain $k$ and the number of firms in each layer $m$. I assume firms estimate statistical properties of each layer of the production network and act probabilistically when making production decisions. This assumption captures the fact that firms, despite not knowing exactly the correlation of risk among producers of their input goods, have some information to estimate their suppliers' risk and make sourcing decisions. In particular, a firm producing good $k$ can, first, infer the distribution of the number $F_{k - 1} \coloneqq \abs{\mathcal{F}_{k - 1}}$ of functioning firms in the previous layer, second select the optimal number $s_{k, i} = \abs{\mathcal{S}_{k, i}}$ of firms from which to supply, and then, third, pick $s_{k, i}$ suppliers with equal probability. The randomness in the first step of the firms' choices arises since all potential suppliers producing good $k - 1$ are ex-ante equal. Hence, each firm in layer $k$ might pick a different set of suppliers, that is, $\mathcal{S}_{k, i}$ is not necessarily equal to $\mathcal{S}_{k, j}$ for some $i$ and $j$, but they do pick the same number, $s_k$, of suppliers:

\begin{equation}
  s_{k, i} = s_k \text{ for all } i.
\end{equation}

If two firms $(k, i)$ and $(k, j)$ pick two sets of suppliers $\mathcal{S}_{k, i}$ and $\mathcal{S}_{k, j}$ with the same size $s_k$, the probability that each of them is functional, $p_{k, i}$ and $p_{k, j}$, is a realization drawn from the same distribution. We can characterise this distribution by looking at the ratio of possible configurations in which, given a random set of suppliers $\mathcal{S}_{k, i}$ and a random set of functioning firms $\mathcal{F}_{k - 1}$, the overlap between them is non-empty. This quantity only depends on the sizes of the two sets, $s_k$ and $F_{k - 1}$. Let $p_{k, i} = P(s_k, F_{k - 1}; m)$ be the probability of the two sets overlapping. This can be written as 

\begin{equation}
  1 - \P\big( \mathcal{S}_{k, i} \cap \mathcal{F}_{k - 1} = \emptyset \big) = P\left(s_k, F_{k - 1}; m\right) =  1 - \overbrace{\binom{m - s_k}{F_{k-1}}}^{\substack{\text{non-overlapping} \\ \text{configurations}}} \Bigg/ \underbrace{\binom{m}{F_{k-1}}}_{\substack{\text{all possible} \\ \text{configurations}}}. 
\end{equation}

This function is symmetric and depends on the layer size $m$ both implicitly, via the support of $F_{k - 1} \in [m]$, and explicitly. Furthermore, $F_{k - 1}$ is a random variable, which implies that $P(s_k, F_{k - 1}; m)$ too is a random variable. We can now make a claim regarding the distribution of $P(s_k, F_{k - 1}; m)$ and its relationship with that of $F_{k - 1}$. To do so, first I show that, if the number of functioning suppliers follows a beta-binomial distribution, then the probability that a downstream firm functions follows a beta distribution (Lemma \ref{lemma:Ftop}). Second, I show that, if the functioning probability of each firm in a layer follows a beta distribution, then the number of firms functioning in that layer follows a beta-binomial distribution (Theorem \ref{theorem:ptoF}). This two results link the distributions of functioning firms of an input good and a downstream good.

\begin{lemma} \label{lemma:Ftop}
  If $F_{k - 1}$ is a beta-binomial r.v. taking values in $[m]$, then \begin{equation*}\lim_{m \rightarrow \infty} P\left(s_k, F_{k - 1}; m\right) \eqqcolon P_k \ \dot{\sim} \ \Beta(1 - \mu_k, \rho_k).\end{equation*} for some parameters $\mu_k$ and $\rho_k$.
  \notes{Because of simulations, I am almost sure this holds true but I can't seem to prove it. Below I give the intuition behind the statement.}
\end{lemma}

The intuition behind Lemma \ref{lemma:Ftop} becomes clear once we enstablish a link between the role the beta distribution plays in the model and the one it plays in bayesian statistics. Whether a firm functions is a binary event $\left\{\mathcal{F}_{k - 1} \cap \mathcal{S}_{k, i} \neq \emptyset \right\}$ with an associated probability, which depends on the size of the two sets $\mathcal{F}_{k - 1}$ and $\mathcal{S}_{k, i}$. If a bayesian observer with a uniform prior seeks to estimate such probability, taking into account the uncertainty around the size $F_{k - 1}$ of the set of functioning upstream firms $\mathcal{F}_{k - 1}$, by observing more and more firms outcomes their posterior converges to a beta distribution (\cite{gelman_bayesian_2004})\footnote{ The problem of inferring the underlying probability of a binary processes in a bayesian manner is the motivation behind the development of the beta-binomial distribution by George Pólya in 1923 (\cite{feller_introduction_1968}) }.


\begin{theorem} \label{theorem:ptoF}
  Consider a r.v. $P_k \sim \Beta(1 - \mu_k, \rho_k)$, for some parameters $(\mu_k, \rho_k)$ and a r.v. defined conditionally on a realisation $p_k$ of $P_k$ as 
  
  \begin{equation}
    \big(F_k \ \vert \ P_k = p_k \big) \sim \Bin(m, p_k).
  \end{equation}

  Then

  \begin{equation}
    F_k \sim \Beta\Bin(m, 1 - \mu_k, \rho_k)
  \end{equation}

  with

  \begin{equation} \label{eq:mean_of_F}
    \E \left[F_k\right] = m \E \left[P_k\right] = m (1 - \mu_k)
  \end{equation}

  and 

  \begin{equation} \label{eq:var_of_F}
    \V \left[F_k\right] = m \mu_k (1 - \mu_k) \big(1 + (m - 1) \rho_k\big),
  \end{equation}

  such that $F_k$ ``inherits'' its parameters from $P$.
\end{theorem}

This theorem follows directly from the definition of the beta-binomial distributionß. Lemma \ref*{lemma:Ftop} and Theorem \ref*{theorem:ptoF} combine into a powerful result: for large $m$, the number of firms that are able to operate $\{F_k\}^{K}_{k = 0}$ at each stage of the production network follows a beta-binomial distribution, each with different parameters $(\mu_k, \rho_k)$. These distributions are fully determined by the initial one, $F_0$, and sourcing choices in each layer, $\{s_k\}^{K}_{k = 0}$. Notice that the initial distribution of functioning firms $F_0$ can be modeled in the same way by simply setting 

\begin{equation}
  F_0 \sim \Beta\Bin(m, 1 - \mu_0, \rho_0)
\end{equation} and taking $\rho_0 \rightarrow 0$ to look at the case in which basal firms have uncorrelated risk.

For agents in downstream layers to solve their maximisation problem, that is, pick a number of suppliers $s_k$, it is sufficient to infer the distribution over the number of functioning firms among their potential suppliers, $F_{k - 1}$. This implies that to track the propagation of risk throughout the supply chain, we can simply look at the of evolution of the two parameters $\mu_k$ and $\rho_k$ through the economy.

\subsection{Interpretation of the Parameters}

It is useful at this point to give an interpretation of $\mu_k$ and $\rho_k$, in the context of our model. Looking back at the relationship between the moments of $F_k$ and these parameters (equations \ref{eq:mean_of_F} and \ref{eq:var_of_F}), we can see that $\mu_k$ is the fraction of firms that are expected to not deliver. Hence, I hereafter refer to $\mu_k$ as \textit{risk}. The parameter $\rho_k$ tracks the degree of correlation in the risk of the firms operating in layer $k$. If $\rho_k \to 0$, then firms' risk is independent, $P_k$ concentrates at $1 - \mu_k$ (red, Figure \ref{fig:distribution-illustration:beta}), and $F_k$ degenerates into a binomial distribution (red, Figure \ref{fig:distribution-illustration:beta-binomial}). On the contrary, if $\rho_k \rightarrow 1$ then firms' risk is perfectly correlated, $P_k$ concentrates at $0$ and $1$ (blue, Figure \ref{fig:distribution-illustration:beta}), and either no firm is able to operate $F_{k} = 0$, with probability $\mu_k$, or all are able to operate $F_{k} = m$, with probability $1 - \mu_k$ (blue, Figure \ref{fig:distribution-illustration:beta-binomial}). Because of this role in increasing the probability of tail events vis-à-vis the binomial distribution, I hereafter refer to $\rho_k$ as \textit{overdispersion}.



\begin{figure}[H]
  \centering
  \begin{subfigure}{.5\textwidth}
    \centering
    \includegraphics[width = \linewidth]{../plots/beta-cdf.png}
    \caption{C.d.f. of $P_k$ for different values of $\rho$}
    \label{fig:distribution-illustration:beta}
  \end{subfigure}%
  \begin{subfigure}{.5\textwidth}
    \centering
    \includegraphics[width = \linewidth]{../plots/betabin-pdf.png} 
    \caption{P.m.f. of $F_k$ for different values of $\rho$}
    \label{fig:distribution-illustration:beta-binomial}
  \end{subfigure}%
  \caption{The two figures show the effect of increasing $\rho$ on the distribution of $P_k$ (left) and $F_k$ (right).}
  \label{fig:distribution-illustration}
\end{figure}

\subsection{Dynamics of Risk}

As argued above, the system is fully described by the evolution of the distribution of functioning firms $\{ F_k \}^{\infty}_{k = 0}$ through the layers, and these are entirely determined by the evolution of the parameters $\mu_k$ and $\rho_k$. Thus, we seek a function $G: [0, 1]^2 \times [m] \to [0, 1]^2$ that maps the parameters of the distribution of the number of functioning firms from one layer to the next:

\begin{equation}
  (\mu_k, \rho_k) = G(\mu_{k - 1}, \rho_{k - 1}; \ s_k).
\end{equation}

This function can be derived by looking at how the distribution in the number of firms $F_{k - 1}$ affects the distribution of $P_k$ and then use the relationship between the moments of $P_k$ and those of $F_k$, layed out in equations \ref{eq:mean_of_F} and \ref{eq:var_of_F} (see Appendix \ref{appendix:derivations} for a more precise derivation). Letting $G_1$ and $G_2$ be the first and second component of $G$ respectively, we can write

\begin{equation} \label{eq:G_mu}
  G_1(\mu, \rho, s) = \left( \mu \  \frac{1 - \rho}{\rho} \right)^{(s)} \Big/ \left(\frac{1 - \rho}{\rho} \right)^{(s)} = \prod^{s - 1}_{n = 0} \frac{\mu + n \frac{\rho}{1 - \rho}}{1 + n \frac{\rho}{1 - \rho}},
\end{equation}

where $\cdot^{(s)}$ is the rising factorial. The expansion in equation (\ref{eq:G_mu}), highlights the effect of the suppliers' correlation in risk propagation. Namely, adding the $n$-th supplier reduces the relative risk by

\begin{equation}
  \frac{G_1(\mu, \rho, n - 1) - G_1(\mu, \rho, n)}{G_1(\mu, \rho, n - 1)} = 1 - \frac{\mu + n \frac{\rho}{1 - \rho}}{1 + n \frac{\rho}{1 - \rho}} = \frac{1 - \mu}{1 + n \frac{\rho}{1 - \rho}}.
\end{equation}

As correlation among suppliers $\rho$ increases, the marginal reduction in risk the firm can expect from adding a supplier decreases (Figure \ref{fig:risk-dumpening}).


\begin{figure}[H]
  \centering
  \includegraphics[width = 0.7\linewidth]{../plots/risk-dumpening.png}
  \caption{Risk dumpening factor $\left(1 + n \frac{\rho}{1 - \rho}\right)^{-1}$}
  \label{fig:risk-dumpening}
\end{figure}

Furthermore, without correlation, each new supplier reduces risk by exactly $(1-\mu)$, such that the dynamical systems converges to the binomial counterpart $\lim_{\rho \rightarrow 0} G_1(\mu, \rho, s) = \mu^s$. Similarly, for sufficiently large $m$, we can derive an analytical expression for the evolution of overdispersion, $\rho$. Letting $\mu' \coloneqq G_1(\mu, \rho, s)$, then

\begin{equation}
  G_2(\mu, \rho, s) = \frac{\left( \mu \  \frac{1 - \rho}{\rho} + s \right)^{(s)} \Big/ \left( \frac{1 - \rho}{\rho} + s \right)^{(s)} - \mu'}{1 - \mu'} = \frac{\prod^{s - 1}_{n = 0} \frac{\mu + (n + s) \frac{\rho}{1 - \rho}}{1 + (n + s) \frac{\rho}{1 - \rho}} - \mu'}{1 - \mu'}.
\end{equation} Again, without suppliers' correlation, the dynamical system converges to the simpler binomial case $\lim_{\rho \rightarrow 0} G_2(\mu, \rho, s) = 0$.

Another intuitive property of the dynamical system is that risk and overdispersion are constant throughout the layers only if the choice firms have a single supplier, $s = 1$, or if the system degenerates, $\mu \in \{0, 1\}$. 

\begin{lemma} \label{lemma:interior_fixed_points}
  In $\mu, \rho \in (0, 1)$, if $s = 1$, every point is a fixed point \begin{equation}
    (\mu, \rho) = G(\mu, \rho, 1). 
  \end{equation}
\end{lemma}

\begin{lemma} \label{lemma:exterior_fixed_points}
  If $s \neq 1$, the only fixed points are the degenerate points $(\mu, \rho)$ in which risk is fully diversified, $(0, 0)$, or there are certain cascading failures, $(1, 0)$.
\end{lemma}

Lemma (\ref{lemma:exterior_fixed_points}) can be proven by looking at the definition of $G_1$ (\ref{eq:G_mu}) and noticing that, for non degenerate values of $\mu$ and $\rho$, $G_1(\mu, \rho, s) \neq \mu$ if $s \neq 1$.

\subsection{Firms' Optimal Diversification}

Now that we characterised a map $G$ of risk $\mu$ and overdispersion $\rho$ between layers given a choice of suppliers $s$, in this section I derive the choice of suppliers that firm make in equilibrium. This choice is obviously contrained to be an integer $s \in \mathbb{N}_{0}$ but I further show that it is useful to consider the extension where $s$ is allowed to take any real positive value $\mathbb{R}_{\geq 0}$. Consider the problem of a firm in layer $k$, that is, based on the risk in the previous layer $(\mu_{k - 1}, \rho_{k - 1})$, choosing a number of suppliers $s_k$ to maximise profits $\Pi(s_k(\mu_{k - 1}, \rho_{k - 1}))$. We can hence write

\begin{equation} \label{eq:firm_problem}
  s_k(\mu_{k - 1}, \rho_{k - 1}) = \arg\max_{s \in \mathbb{N}_0} \Pi(s) = \arg\max_{s \in \mathbb{N}_0} \left\{ \pi \Big(1 - G_1(\mu_{k - 1}, \rho_{k - 1}, s)\Big) - \kappa \  s \right\}.
\end{equation}



The function $(\mu, \rho) \mapsto G(\mu, \rho, s)$ has interior fixed points only if $s = 1$, even if $s$ is allowed to take non-integer values (i.e. Lemma \ref{lemma:interior_fixed_points} and \ref{lemma:exterior_fixed_points} hold over $\Re_{\geq 0}$ not only over $\mathbb{N}_0$). Furthermore, the qualitative behaviour of the vector field $G$ changes only if $s + \varepsilon$ and $s$ have different integer part (see Appendix \notes{todo}). 

\begin{lemma}
  The distance between $\tilde{s}_k$ and $s_k$ is strictly smaller than 1.
\end{lemma}

This lemma follows from the fact that $\Pi(s)$ is strictly concave and admits a unique maximum\footnote{Since $\Pi$ admits a maximum and is strictly concave, it is guaranteed that a value $\bar{s}$, with the property that $\Pi(\bar{s}) = \Pi(\bar{s} + 1)$ exists. Both the maximum over the real and the integers are guaranteed then to be in the interval $\tilde{s}_k, s_k \in [\bar{s}, \bar{s} + 1]$. Hence, $\abs{\tilde{s}_k - s_k} < 1$.}. These results guarantee that the properties of the extended dynamical system over the real numbers transfers over to that determined by integers $s_k$. Hence, for analytical convenience, letting

\begin{equation}
  \tilde{s}_k(\mu_{k - 1}, \rho_{k - 1}) \coloneqq \arg\max_{s \in \mathbb{R}_{\geq 0}} \Pi(s),
\end{equation} I hereafter focus on the map

\begin{equation}
  \tilde{G}(\mu, \rho) \coloneqq G(\mu, \rho, \tilde{s}(\mu, \rho)),
\end{equation} and explicitly refer to the difference with the integer dynamical system $G(\mu, \rho, s(\mu, \rho))$ when these arise. The function $\tilde{s}_k$ is implicitly determined by the first order condition on $\Pi(s)$, namely

\begin{equation} \label{eq:exact-foc}
  \begin{split}
    -\frac{\kappa}{\pi} &= \left. \frac{\partial G_1}{\partial s} \right\vert_{(\mu, \rho, \tilde{s})}  \\
    &= \tilde{G}_1(\mu, \rho) \left(\psi_0\left(\tilde{s}(\mu, \rho) + \mu \frac{1 - \rho}{\rho} \right) - \psi_0\left(\tilde{s}(\mu, \rho) + \frac{1 - \rho}{\rho} \right)  \right),
  \end{split}
\end{equation}

where $\psi_0$ is the digamma function, which, for small values of $\rho$, can be approximated as $\psi_0(x) \sim \log x - \frac{1}{2x}$. Looking at the countour plot of $\tilde{s}_k$ (Figure \ref{fig:agents-optimum}) it is clear how the interplay between risk $\mu_{k - 1}$ and suppliers' correlation $\rho_{k - 1}$ affect the choice of the firm in layer $k$. Particularly, given a level of risk in the previous layer $\mu_{k - 1}$, an increase in correlation among suppliers in the previous layer $\rho_{k - 1}$ disincentivises firms' diversification. This already hints at an important externality that endogenous firm decisions imposes on the production network. The firm response to correlation of suppliers is to underdiversify, while the social optimum would be to diversify in order to reduce such correlation. This misalignment of incentives arises because the firm does not reap the downstream benefits of underdiversification. Another important takeaway from the function $\tilde{s}_k$ is that, as average risk among suppliers $\mu_{k - 1}$ increases (i.e. moving left in the contourf plot), there is a level of risk after which firms are better off not prodcuing and stop sourcing goods $\tilde{s}_k = 0$.

\begin{figure}[H]
  \centering
  \includegraphics[width = \linewidth]{../plots/agents.png}
  \caption{Contour plot of $\tilde{s}_k(\mu, \rho)$ and $\pi = 10 \kappa$. Highlighted the steady state manifold, $(\bar{\mu}, \bar{\rho})$: $\tilde{s}_k(\bar{\mu}, \bar{\rho}) = 1$.}
  \label{fig:agents-optimum}
\end{figure}

\iffalse
  Furthermore, $\bar{s}$ be such that\footnote{Since $\Pi$ admits a maximum and is strictly concave, $\bar{s}$ is guaranteed to exist and be unique.} $\Pi(\bar{s}) = \Pi(\bar{s} + 1)$. Then we know that 

  \begin{equation}
    s_{k + 1} \in \big\{ \ceil{\bar{s}}, \floor{\bar{s} - 1} \big\} \cap \big\{ \ceil{\tilde{s}}, \floor{\tilde{s}} \big\}
  \end{equation}

  This condition implies that $\abs{s(\mu_k, \rho_k) - \tilde{s}(\mu_k, \rho_k)} < 1$. Given the smoothness of $s \mapsto G(\mu, \rho, s)$, the qualitative properties of the dynamical systems

  \begin{equation*}
      G(\mu_k, \rho_k, s_{k + 1}(\mu_k, \rho_k)) \text{ and } G(\mu_k, \rho_k, \tilde{s}_{k + 1}(\mu_k, \rho_k))
  \end{equation*}

  are be similar.

  \notes{Yet another completly unmotivated claim! Try and use the shadowing lemma. For an initial condition $x_0$, there exist an orbit of $\tilde{G}$ arbitraily ($\delta$-$\epsilon$) close to that of $G$.}

 % TODO: Move to appendix
  \begin{figure}[H]
    \centering
    \includegraphics[width = 0.7\linewidth]{../plots/convex-integer-optimization.png}
    \caption{Relationship between $s_{k + 1}, \tilde{s}$, and $\bar{s}$. Plot with $\mu_k = 0.115$ and $\rho_k = 0.06$.}
    \label{fig:convex-integer-optimization}
  \end{figure}
\fi

\subsection{A Special Case: No Correlation Risk}

It is instructive to first consider the dynamics of risk in the case where there is no correlation among suppliers. That is, when the function $G$ converges to its familiar binomial counterpart,

\begin{equation}
   \lim_{\rho \rightarrow 0} \tilde{G}(\mu, \rho) = \begin{pmatrix} \mu^{\tilde{s}(\mu, 0)} \\ 0 \end{pmatrix} 
\end{equation}

First, the degenerate dynamical system characterised by the map $\tilde{g} \coloneqq \lim_{\rho \to 0} \tilde{G}_1$ allows us derive stronger analytical results and properties that can be later generalised to the map $\tilde{G}$ with $\rho > 0$. Second, it links our model to the case of simple goods studied by \citein{elliott_supply_2022}. Despite studying conceptually different problems (their model deals with edge percolation and investment in relationship), both models admit parameter regions with similar interpretations and cascading failures.

In the case in which there is no overdispersion, the first order condition on the firms' optimisation problem on the choice of the number of suppliers $\tilde{s}(\mu_k, 0)$ \footnote{
  The first order condition remains valid in the limit, given the property
  \begin{equation}
    \lim_{\rho \rightarrow 0} \frac{\partial}{\partial s}G_1 = \lim_{\rho \rightarrow 0} G_1 \times \left(\psi_0\left(s + \mu \frac{1 - \rho}{\rho} \right) - \psi_0\left(s + \frac{1 - \rho}{\rho} \right) \right) = \mu^{s} \log(\mu) = \frac{\partial}{\partial s} \lim_{\rho \rightarrow 0} G_1
  \end{equation}
} is

\begin{equation} \label{eq:one-dimension:foc}
  \mu_k^{\tilde{s}(\mu_k, 0)} \log(\mu_k) = -\frac{\kappa}{\pi}.
\end{equation} which yields the evolution of risk

\begin{equation}
  \mu_{k+1} = \tilde{g}(\mu_k) = -\frac{\kappa / \pi}{\log(\mu_k)}.
\end{equation}

Plotting the function $\tilde{g}$ (Figure \ref{fig:one-dimensional:cobweb}) we can immediately see that if profits $\pi$ are sufficiently large relative to the pairing costs $\kappa$, most levels of basal risk $\mu_0$ are diversified away by downstream firms and there are no cascading failures, namely $\mu_k < 1$ for all $k$. In this low relative cost parameter region the system has two steady states, a low risk stable steady state (blue) and a high risk unstable steady state (red). The stability condition of the fixed points,

\begin{equation}
  \mu \pi < \kappa,
\end{equation}

immediately highlights the qualitative difference between them. At the stable fixed point the ``value at risk'', $\mu \pi$, is smaller than the marginal diversification cost. Hence, it is optimal for the firms to respond to an increase in risk, $\mu$, by diversifying. On the other hand, at the unstable fixed point, the optimal response to an increase in risk is to cease production.

\begin{figure}[H]
  \centering
  \includegraphics[width = 0.8\textwidth]{../plots/one-dim-cobweb.png}
  \caption{The function $\tilde{g}$ for low (orange) and high (green) profits.}
  \label{fig:one-dimensional:cobweb}
\end{figure}

As seen in (Figure \ref{fig:one-dimensional:trajectory}), there is a level of relative costs $\kappa / \pi$ at which, no matter the initial level of risk, firms do not diversify it. In basal industries this effect might appear negligible but risk compounds quickly and and arbitrary small increase in diversification costs $\kappa$ can lead to downstream cascading failures. \notes{Expand section}


\begin{figure}[H]
  \centering
  \includegraphics[width = 0.8\textwidth]{../plots/one-dim-trajectory.png}
  \caption{The propagation of risk $\mu_k$. A small decrease in relative profits can lead to cascading failures. }
  \label{fig:one-dimensional:trajectory}
\end{figure}

A natural question is, at what level of $\kappa / \pi$ do we observe such change in firms' behaviour (as seen in Figure \ref{fig:one-dimensional-bifurcation})? It is not hard to see that this occurs if 

\begin{equation}
  \kappa / \pi > e^{-1}.
\end{equation}

\notes{Linked to the secretary problem?}

\begin{figure}[H]
  \centering
  \includegraphics[width = 0.85\linewidth]{../plots/one-dim-bif.png}
  \caption{Bifurcation diagram of the limit system $\mu_{k + 1} = \tilde{g}(\mu_k)$.}
  \label{fig:one-dimensional-bifurcation}
\end{figure}

\subsection{Introducing correlation risk}

In this section I extend the study of the dynamics of risk $\tilde{G}$ to the case where there is correlation in the production network, that is $\rho_k > 0$ for some $k$. First Lemmas (\ref{lemma:interior_fixed_points}) and (\ref{lemma:exterior_fixed_points}) guarantee that the fixed points $(\bar{\mu}, \bar{\rho})$ of $\tilde{G}$ need to satisfy $\tilde{s}_k(\bar{\mu}, \bar{\rho})= 1$. Using this condition in the implicit definition of $\tilde{s}_k$ (\ref{eq:exact-foc}) allows us to write the manifold of fixed points as

\begin{equation}
  -\frac{\kappa}{\pi} = \left( \psi_0\left(1 - \bar{\mu} + \frac{ \bar{\mu} }{\bar{\rho}} \right) - \psi_0\left(\frac{1}{\bar{\rho}} \right) \right) \bar{\mu},
\end{equation}

which can be approximated, for small $\bar{\rho}$, as

\begin{equation}
  -\frac{\kappa}{\pi} = \bar{\mu}\log(\bar{\mu}) + \bar{\rho} \left( \bar{\mu}^2 - \frac{1}{2} \bar{\mu} + \frac{1}{2} \right)
\end{equation}



\begin{figure}[H]
  \centering
  \includegraphics[width = \linewidth]{../plots/basin_small.png}
  \caption{Basin of attraction of $\tilde{G}$ with $\pi = 10 \kappa$. The color of a point indicates the value $1 - \bar{\mu}$ of the attractor of that point.}
  \label{fig:two-dimensional:basin}
\end{figure}

\begin{figure}[H]
  \centering
  \includegraphics[width = \linewidth]{../plots/bifurcation.png}
  \caption{Bifurcation diagram }
  \label{fig:two-dimensional:bifurcation}
\end{figure}



\end{document}