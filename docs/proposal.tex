\documentclass[american, abstract=on]{scrartcl}

    \newcommand{\lang}{en}

    \usepackage{babel}
    \usepackage[utf8]{inputenc}

    \usepackage{csquotes}

    \usepackage{amsmath, amssymb, mathtools, bbm}
    \usepackage{xcolor}
    \usepackage{xcolor-solarized}
    \usepackage{bm}


    \usepackage{graphicx}
    \usepackage{wrapfig}
    \usepackage{relsize}
    \usepackage{makecell}
    \usepackage{booktabs}
    \usepackage[font=footnotesize,labelfont=bf]{caption}
    \usepackage{subcaption}
    \usepackage{float}
    \usepackage{multirow} 
    
    % Diagrams
    \usepackage{tikz} 
    \usepackage{tikzit}
    \usetikzlibrary{fit}
    \input{diagrams/percolation.tikzstyles}

    \newcommand{\inputTikZ}[2]{%  
      \scalebox{#1}{\input{#2}}  
    }
    
    % Refs
    \usepackage{hyperref}
    \usepackage{cleveref}
    \hypersetup{
        colorlinks = true, 
        urlcolor = blue,
        linkcolor = blue, 
        citecolor = blue 
      }      

    \usepackage{subfiles} % Load last

    % Paths

    % Formatting
    \setlength{\parindent}{0em}
    \setlength{\parskip}{0.5em}
    \setlength{\fboxsep}{1em}
    \newcommand\headercell[1]{\smash[b]{\begin{tabular}[t]{@{}c@{}} #1 \end{tabular}}}

    % Graphs

    % Math commands

    \newcommand{\diff}{\text{d}}
    \renewcommand{\Re}{\mathbb{R}}
    \newcommand{\C}{\mathcal{C}}
    \newcommand{\F}{\mathcal{F}}
    \newcommand{\X}{\mathcal{X}}
    \newcommand{\G}{\mathcal{G}}
    \newcommand{\N}{\mathcal{N}}

    \newcommand{\uI}[2][s]{\int^1_0 #2 \ \text{d} #1}
    \newcommand{\uH}[2][s]{\int^\frac{1}{2}_0 #2 \ \text{d} #1}
    \newcommand{\uF}[2][s]{\int^1_\frac{1}{2} #2 \ \text{d} #1}
    \newcommand{\norm}[1]{\left\lVert#1\right\rVert}

    % Bibliography

    \usepackage[bibencoding=utf8, style=apa]{biblatex}
    \bibliography{supply-chain-reallocation}

    \newcommand{\citein}[1]{\citeauthor{#1} (\citeyear{#1})}

    \newcommand\notes[1]{\textcolor{teal}{\textbf{#1}}}
    \newcommand\red[1]{\textcolor{red}{#1}}

    % Make title page

    \author{Andrea Titton}
    \title{Limited Information and Fragility of Endogenous Production Networks}
    
\begin{document}

\maketitle

\section{Introduction}

It has long been recognized that increasingly complex production networks, by allowing the production of more technically sophisticated goods and fostering specialisation, are a driver of economic growth (\cite{kremer_population_1993,acemoglu_endogenous_2020}). At the same time, highly complex production networks increase the probability of cascading failures, thereby exacerbating fluctuations caused by idiosyncratic shocks (\cite{acemoglu_network_2012,baqaee_macroeconomic_2019,guerrieri_macroeconomic_2020}). In light of this, understanding how strategic firms' decisions, regarding suppliers and investments, endogenously determine the architecture of production networks is necessary to detect fragility and for policy intervention. 

In this paper, I develop a model in which firms endogenously pick suppliers in order to minimise the risk of production failures. In line with the literature (\cite{elliott_supply_2022}), I show that, under complete information, firms do not internalise the downstream cost of their production failures, such that their decisions induce a greater aggregate risk than the socially optimum. On the other hand, when firms have limited information on the structure of the production network beyond their immediate supplier, they tend to over-diversify, thereby bringing the production network closer to the social optimum. 

To understand this result, consider the two simple production networks illustrated in Figure \ref{fig:example}. Firm $1$ produces the yellow good and needs to source the red good from either firm $2$ and $3$ or from both. In turn, these two firms source the blue good from $4$ and $5$. If $4$ supplies $2$ and $5$ supplies $3$ (\ref{fig:example:idio}), firm $1$ can diversify its inputs by supplying from both firm $2$ and $3$, since their upstream risk is idiosyncratic. On the other hand, if $4$ supplies both $2$ and $3$ (\ref{fig:example:cov}), the upstream risk of the two suppliers is covariate, hence firm $1$ cannot diversify its inputs.

\begin{figure}[ht]
  \centering
  \begin{subfigure}{.5\textwidth}
    \centering
    \inputTikZ{0.7}{diagrams/example-idio.tikz} 
    \caption{Idiosyncratic upstream risk}
    \label{fig:example:idio}  
  \end{subfigure}%
  \begin{subfigure}{.5\textwidth}
    \centering
    \inputTikZ{0.7}{diagrams/example-covariate.tikz} 
    \caption{Covariate upstream risk}
    \label{fig:example:cov}
  \end{subfigure}
  \caption{Two production networks. Firm $4$ and $5$ produce a blue good, which is used as input for the red good, produced by $2$ and $3$, which is used by firm $1$ to produce the yellow good. The arrow represent realised supplier relations and the dashed line represents potential supplier relations. Firm $1$ must hence choose between supplying from $2$ or $3$. The left with idiosyncratic supplier risk and the right with covariate supplier risk.}
  \label{fig:example}
\end{figure}

Now assume that firm $1$ cannot observe the supplier decisions of firm $2$ and $3$. \notes{What happens in this case}

\begin{figure}[ht]
  \centering
  \inputTikZ{0.7}{diagrams/example-limited.tikz} 
  \caption{Same production network as in Figure \ref{fig:example}, but with limited information.}
  \label{fig:example:unknown}  
\end{figure}
\section{Model}

\subsection{Goods and firms}

In the economy there are $n$ firms $\N = \{1, 2, 3, 4, \ldots, n \}$. Each firm produces a unique good, such that the goods can be thought of as a partition of $\N$. Let the set of goods be

\begin{equation*}
    \G = \{ \overbrace{\{1, 2, 3, \ldots, n_a\}}^{a},  \overbrace{\{n_a + 1, n_a + 2, \ldots, n_b\}}^{b}, \ldots \} 
\end{equation*}

such that we can write $i \in g \in \mathcal{G}$ if $i$ produces $g$.

% --- Bibliography
\newpage
\nocite{*}
\printbibliography

\end{document}