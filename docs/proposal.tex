\documentclass[american, abstract=on]{scrartcl}

    \newcommand{\lang}{en}

    \usepackage{babel}
    \usepackage[utf8]{inputenc}

    \usepackage{csquotes}

    \usepackage{amsmath, amssymb, mathtools, bbm}
    \usepackage{xcolor}
    \usepackage{xcolor-solarized}
    \usepackage{bm}

    \usepackage{graphicx}
    \usepackage{wrapfig}
    \usepackage{relsize}
    \usepackage{makecell}
    \usepackage{booktabs}
    \usepackage[font=footnotesize,labelfont=bf]{caption}
    \usepackage{subcaption}
    \usepackage{float}
    \usepackage{multirow} 
    
    % Diagrams
    \usepackage{tikz} 
    \usepackage{tikzit}
    \usetikzlibrary{positioning,fit,calc}
    \input{diagrams/percolation.tikzstyles}

    \newcommand{\inputTikZ}[2]{%  
      \scalebox{#1}{\input{#2}}  
    }
    
    % Refs
    \usepackage{hyperref}
    \usepackage{cleveref}
    \hypersetup{
        colorlinks = true, 
        urlcolor = blue,
        linkcolor = blue, 
        citecolor = blue 
      }      

    \usepackage{subfiles} % Load last

    % Paths

    % Formatting
    \setlength{\parindent}{0em}
    \setlength{\parskip}{0.5em}
    \setlength{\fboxsep}{1em}
    \newcommand\headercell[1]{\smash[b]{\begin{tabular}[t]{@{}c@{}} #1 \end{tabular}}}

    % Graphs

    % Math commands

    \newcommand{\diff}{\text{d}}
    \renewcommand{\Re}{\mathbb{R}}
    \newcommand{\C}{\mathcal{C}}
    \newcommand{\F}{\mathcal{F}}
    \newcommand{\X}{\mathcal{X}}
    \newcommand{\G}{\mathcal{G}}
    \newcommand{\I}{\mathcal{I}}
    \newcommand{\N}{\mathcal{N}}
    \newcommand{\PF}{\mathcal{P} \F}

    \renewcommand{\P}{\mathbb{P}}
    \newcommand{\E}{\mathbb{E}}

    \newcommand{\uI}[2][s]{\int^1_0 #2 \ \text{d} #1}
    \newcommand{\uH}[2][s]{\int^\frac{1}{2}_0 #2 \ \text{d} #1}
    \newcommand{\uF}[2][s]{\int^1_\frac{1}{2} #2 \ \text{d} #1}
    \newcommand{\norm}[1]{\left\lVert#1\right\rVert}
    \newcommand{\abs}[1]{\left\lvert#1\right\rvert}

    % Bibliography

    \usepackage[bibencoding=utf8, style=apa, backend=biber]{biblatex}
    \addbibresource{supply-chain-reallocation.bib}

    \newcommand{\citein}[1]{\citeauthor{#1} (\citeyear{#1})}

    \newcommand\notes[1]{\textcolor{teal}{\textbf{#1}}}
    \newcommand\red[1]{\textcolor{red}{#1}}

    % Make title page

    \author{Andrea Titton}
    \title{Limited Information and Fragility of Endogenous Production Networks}
    
\begin{document}

\maketitle

\section{Introduction}

It has long been recognized that increasingly complex production networks drive economic growth by allowing the production of more technically sophisticated goods and fostering specialisation (\cite{acemoglu_endogenous_2020}). At the same time, highly complex production networks increase the probability of cascading failures, thereby exacerbating fluctuations caused by idiosyncratic shocks (\cite{baqaee_macroeconomic_2019}). In light of this, endogenising the supplier decisions of firms is central in understanding the opportunities and threats posed by complex production networks. Recent work by \citein{elliott_supply_2022} shows how endogenous production networks can organise towards fragility, when firms operate under uncertainty. This is an important result but it does not account for the fact that firms have limited information about their supplier risk, particularly further up the production network (\cite{pwc_supply_2019}). 

To address this, I develop a model in which firms endogenously pick suppliers in order to minimise the risk of production failures. First, in line with \citein{elliott_supply_2022}, I show that, under complete information, firms do not internalise the downstream cost of their production failures, such that their decisions induce a greater aggregate risk than the social optimum. Second, I show that this result gives rise to fundamental non-linearities in the firms suppliers' decisions which do not ``smooth out'' once aggregated at the production network level. Third, when firms have limited information on the structure of the production network beyond their immediate suppliers, the degree of diversification will depend on their beliefs on the covariance of the risk of the potential suppliers. I show that, if firms learn over time the distribution of the risk by observing their suppliers, their choices bring the production network closer to the social optimum.

\subsection{Example}

To understand these results, consider the two simple production networks illustrated in Figure \ref{fig:example}. Firm \textit{one} produces the yellow good and needs to source the red good from either firm \textit{two}, \textit{three}, or both. In turn, these two firms source the blue good from \textit{four} and \textit{five}. Assume that sourcing from a firm requires a fixed cost. If producers of the red good have different suppliers (\ref{fig:example:idio}), firm \textit{one} can diversify its inputs by supplying from both firm \textit{two} and \textit{three}, since their upstream risk is idiosyncratic. On the other hand, if the producers of the red good have the same supplier, hence covariate risk (\ref{fig:example:cov}), the reduction in risk that firm \textit{one} would obtain by diversifying might be too small to justify paying the necessary fixed cost. If there are firms downstream using the yellow good as input, this choice by \textit{one} would increase the fragility of the production network. ``Under-diversification'' occurs because \textit{one} is not compensated for the reduced risk across the supply chain.

\begin{figure}[H]
  \centering
  \begin{subfigure}{.5\textwidth}
    \centering
    \inputTikZ{0.5}{diagrams/example-idio.tikz} 
    \caption{Idiosyncratic upstream risk}
    \label{fig:example:idio}  
  \end{subfigure}%
  \begin{subfigure}{.5\textwidth}
    \centering
    \inputTikZ{0.5}{diagrams/example-covariate.tikz} 
    \caption{Covariate upstream risk}
    \label{fig:example:cov}
  \end{subfigure}
  \caption{Two production networks. The left with idiosyncratic supplier risk and the right with covariate supplier risk. Firm \textit{one} needs to pick a red good supplier.}
  \label{fig:example}
\end{figure}

Now assume that \textit{one} cannot observe the supplier decisions of the red good producers but assigns equal probability to every possible configuration. Given that there are more configurations in which diversification is optimal\footnote{
  Diversification is optimal if \inputTikZ{0.1}{diagrams/conf-idyof.tikz} or, with equal probability, \inputTikZ{0.1}{diagrams/conf-covpref.tikz}, and not optimal if \inputTikZ{0.1}{diagrams/conf-covind.tikz}
}, diversification under incomplete information and with uniform priors will be closer to the social optimum. This simple example makes it immediately clear that firms beliefs play a crucial role in the endogenous formation of production networks. 

\begin{figure}[H]
  \centering
  \inputTikZ{0.5}{diagrams/example-limited.tikz} 
  \caption{Same production network as in Figure \ref{fig:example}, but with limited information. Firm $1$ cannot observe edges stemming from opaque nodes.}
  \label{fig:example:unknown}  
\end{figure}

\section{Literature review}

The model presented here introduces insights from the literature of games on networks and learning on networks to the macroeconomic literature on endogenous production network.

Particularly relevant are the models developed by \citein{dasaratha_bayesian_2018} and \citein{dasaratha_learning_2021} where agents observe neighbours characteristics to learn about a latent network state.   

\notes{TODO: I collected the relevant papers, need to complete this section}

\section{Model}

\subsection{Goods and firms}

In the economy there are $n$ firms $\N = \{1, 2, 3, 4, \ldots, n \}$. Each firm produces a unique good, such that the goods can be thought of as a partition of $\N$. Namely, let the set of goods be

\begin{equation}
    \G = \{ \overbrace{\{1, 2, 3, \ldots, n_a\}}^{a},  \overbrace{\{n_a + 1, n_a + 2, \ldots, n_b\}}^{b}, \ldots \},
\end{equation}

then we can write $i \in g$ with $g \in \G$ if $i$ produces $g$. Only a random subset of the firms, $\F \subseteq \N$, is functional and hence able to produce and supply goods. The probability with which a firm is functional depends on an idiosyncratic component and on the choice of its suppliers. 

\subsection{Supplier Decisions and Production}

Each good $g$ requires a set of inputs to be produced, $\mathcal{I}(g) \subseteq \mathcal{G}$. Before observing the set of functional firms $\F$, each firm $i \in g$ needs to pick, for each input good $s \in \mathcal{I}(g)$, a set of suppliers producing that good, denoted $x_i^s \subseteq s$. Establishing a relation with a supplier has a fixed cost $\kappa$. If, among the suppliers of a necessary input, none are functional (that is, $x_i^s \cap \F$ is empty), then firm $i$ cannot be functional. Hence the fundamental trade-off that firms face is between paying the cost of diversifying and its added benefit to the probability of being functional. Hereafter, I will say that a firm $i \in g$ is ``potentially'' functional if it satisfies $x_i^s \cap \F$ is not empty for all $s \in \I(g)$. Given a set of suppliers $x_i$ for each firm $i$, we can let $\PF$, with $\F \subseteq \PF \subseteq \N$, be the set of potentially functional firms.

Going back to the example production network introduced above, consider the case in which firm \textit{four} is not functional, illustrated in Figure \ref{fig:functional_example}. This immediately implies that firm \textit{two} is not functional. Hence, firm \textit{one} would be functional only if it had decided to diversify its input and pay $2\kappa$ fixed costs (\ref{fig:functional_example:yes}) instead of sourcing exclusively from \textit{two} and paying simply $\kappa$ (\ref{fig:functional_example:no}).

\begin{figure}[H]
  \centering
  \begin{subfigure}{.5\textwidth}
    \centering
    \inputTikZ{0.5}{diagrams/example-functional.tikz} 
    \caption{Functional $1$}
    \label{fig:functional_example:yes}  
  \end{subfigure}%
  \begin{subfigure}{.5\textwidth}
    \centering
    \inputTikZ{0.5}{diagrams/example-notfunctional.tikz} 
    \caption{Not functional $1$}
    \label{fig:functional_example:no}
  \end{subfigure}
  \caption{Production network of Figure \ref{fig:example}, where $\square$ represents a non-functional firm.}
  \label{fig:functional_example}
\end{figure}

\subsection{Conditional and Unconditional Probability of Being Functional}

As mentioned above, there are two sources of uncertainty on whether a given firm is functional: an idiosyncratic ``private'' risk and the choice of its suppliers\footnote{Note that $\F \subseteq \PF$ implies that $\P\Big(i \in \F \Big) = \P\Big(i \in \F \ \vert \ i \in \PF \Big) \times \P\Big( i \in \PF \Big)$}. The former is determined by an heterogeneous firm risk measure $\mu_i$, such that  

\begin{equation}
  \P\Big(i \in \F \ \vert \ i \in \PF \Big) = 1 - \mu_i.
\end{equation}

The latter depends on the probability of the chosen suppliers being functional, namely

\begin{equation}
  \P\Big( i \in \PF \Big) = \P\Big( \forall s \in \I(g): \ x_i^s \cap \F \neq \emptyset \Big).
\end{equation}

The risk associated with the choice of suppliers can be idiosyncratic or covariate. The former is guaranteed \textit{ex ante} when the supply chain structure presents no cycles. In this case, the supplier risk probability can be written as 

\begin{equation}
  \P\Big( i \in \PF \Big) = \prod_{s \in \I(g)} 1 - \P\Big( x^s_i \cap \F = \emptyset\Big).
\end{equation}

\subsection{Payoffs}

Firm operate for an exogenous payoff $\pi_i$\footnote{This can be easily microfounded using an endogenous production network model, as in \citein{acemoglu_endogenous_2020}.} and pay a cost $\kappa$ of establishing relationships with a supplier. Hence, the expected payoff of the firm can be written as

\begin{equation}
  \Pi(x_i) = \pi_i \ \P\Big( i \in \F \Big) - \kappa \sum_{s \in \I(g)} \abs{x^s_i}.
\end{equation}

\section{Vertical economy} \label{sec:vertical}

\subsection{Ignoring correlations in suppliers' risk} \label{sec:vertical:ignoring}

Consider a vertical economy (Figure \ref{fig:vertical}) with $n$ goods

\begin{equation}
  \G = \{g_0, g_1, g_2, \ldots, g_n\}.
\end{equation}

Each good $g_i$ is produced by $m_i$ firms and firms producing the same good have the same payoff $\pi_i$ and risk $\mu_i$.

\begin{figure}[H]
  \centering
  \inputTikZ{0.7}{diagrams/vertical-economy.tikz} 
  \caption{One possible realization of a vertical economy with three goods}
  \label{fig:vertical}
\end{figure}

If firms cannot observe the realisation of their suppliers' network, ex ante, all potential suppliers are equivalent from a firms' point of view. Hence, the problem of a firm $f_i$ producing $g_i$ reduces to picking the size of its suppliers $S_i = \abs{x_{f_i}}$ where $x_{f_i} \subseteq g_{i - 1}$. Once the size is chosen, each supplier will be chosen with equal probability. We can then write the probability of $f_i$ being functional recursively as

\begin{alignat*}{2}
  \P\Big( f_i \in \F \Big) &= \ &&\overbrace{\P\Big( f_i \in \F \ \vert \ f_i \in \PF \Big)}^{= 1 - \mu_i} \ \P\Big( f_i \in \PF \Big) \\
  &+ &&\underbrace{\P\Big( f_i \in \F \ \vert \ f_i \notin \PF \Big)}_{= 0} \  \P\Big( f_i \notin \PF \Big) \\
  &= &&(1 - \mu_i) \left( 1 - \P\Big( f_{i-1} \notin \F \Big)^{S_i} \right) \\
  &= &&(1 - \mu_i) \left( 1 - \left(1 - \P\Big( f_{i-1} \in \F \Big)\right)^{S_i} \right)
\end{alignat*}

Renaming $p_i \coloneqq \P\Big( f_i \in \F \Big)$ we have the recursive definition

\begin{equation}
  \begin{split}
    p_{i} &= (1 - \mu_i) \left( 1 - (1 - p_{i-1})^{S_i} \right) \\
    \text{with } p_0 &= (1 - \mu_0).
  \end{split}
\end{equation}


\subsubsection{Problem of the firm}

Given this recursive definition, the problem of firm $f_i$ can be formulated as having to choose an $S_i \in \mathbb{Z}$ to maximise

\begin{equation}
  \Pi_i(S_i) = \pi_i \ p_i(S_i) - \kappa \ S_i.
\end{equation}

The optimal $S_i$ will be such that $S_i + 1$ yields lower marginal benefits than marginal costs, that is,

\begin{equation}
  \begin{split}
    (1 - p_{i-1})^{S_i} - (1 - p_{i-1})^{S_i + 1} &\leq \frac{\kappa}{\pi_i \ (1 - \mu_i)} \\
    (1 - p_{i-1})^{S_i} p_{i - 1} &\leq \frac{\kappa}{\pi_i \ (1 - \mu_i)} \\
    S_i&\leq \frac{\log\left( \kappa / \pi_i \right) - \log(1 - \mu_i) - \log p_{i - 1}}{\log(1 - p_{i-1})}
  \end{split}
\end{equation}

\begin{figure}[H]
  \centering
  \includegraphics[width = \textwidth]{plots/splot.pdf} 
  \caption{Optimal number of suppliers of $f_i \in g_i$ as a function of $p_{i - 1}$ for different levels of $\kappa \Big/ \pi_i (1 - \mu_i)$}
  \label{fig:splot}
\end{figure}

\begin{figure}[H]
  \centering
  \includegraphics[width = \textwidth]{plots/pplot.pdf} 
  \caption{Induced probability $p_i$ given $p_{i - 1}$ for different $\mu_i$, assuming $\kappa / \pi_i = 1/20$}
  \label{fig:pplot}
\end{figure}

\subsubsection{Social optimum}

Consider the social planner problem of maximizing the total firms payoff. Letting $S = \begin{pmatrix} S_1 & S_2 & \ldots & S_n \end{pmatrix}$ be the sequence of suppliers size, the problem of the firm can be formulated as

\begin{equation}
  \max_{S \in \mathbb{Z}^n} W(S) = \max_{S \in \mathbb{Z}^n} \sum^n_{i = 0} m_i \ \Pi_i(S).
\end{equation}

We can compute the first order condition\footnote{The function $W(S)$ is concave, hence the maximum over $\mathbb{Z}^n$ is one of the vertices of the cube containing the maximum over $\Re^n$.} with respect to the number of suppliers of a layer $j$ using the fact that downstream firms' decision to not affect the risk of upstream firms, that is, $\frac{\partial p_i}{\partial S_j} = 0$ if $i < j$. This condition yields

\begin{equation}
  \underbrace{\sum_{i > j} m_i \ \pi_i \ \frac{\partial p_i}{\partial S_j}}_{\text{downstream externality}} + m_j  \underbrace{\left( \pi_j \frac{\partial p_j}{\partial S_j} - \kappa\right)}_{\text{firm's optimization}} = 0.
\end{equation}

We can see the role of the externality by writing the optimality condition as 

\begin{equation} \label{eq:distortion}
  \overbrace{\Bigg(\underbrace{\sum_{i> j} \frac{m_i\pi_i}{m_j\pi_j} \ E_{i, j}(S)}_{\text{downstream externality}} + \ 1 \Bigg)\ \frac{\partial p_j}{\partial S_j}(S)}^{\text{marginal benefit}} = \frac{\kappa}{\pi_j}
\end{equation}

with

\begin{equation}
  E_{i, j}(S) = S_i \ (1 - \mu_i) (1 - p_{i-1})^{-1 + S_i}\ E_{i - 1, j}(S).
\end{equation}

An increase in the number of suppliers of a firm has a strictly positive externality on downstream firms, $E_{i, j}(S) > 0$. Hence, the downstream externalities increases the marginal benefit curve, which yields a higher number of suppliers for each, but the last, layer, as opposed to the competitive equilibrium (as displayed in Figure \ref{fig:vert_foc}).


\begin{figure}[H]
  \centering
  \includegraphics[width = \textwidth]{plots/vert_foc.pdf} 
  \caption{The role of distortion in the }
  \label{fig:vert_foc}
\end{figure}

\subsection{Considering correlation in suppliers' risk} \label{sec:vertical:considering}

The solution in section \ref{sec:vertical:ignoring} is valid only as $m_i$ is sufficiently large so that correlations between firms' risk, caused by having shared suppliers, is negligible. Now I will turn my attention to the case where the correlation of suppliers risk is non-negligible. 

For illustration purposes consider a vertical economy with three goods,

\begin{equation}
  \G = \{g_0, g_1, g_2\}
\end{equation}

and with idiosyncratic risk only in the basal node, $\mu_0 = \mu > 0$ and $\mu_1 = \mu_2 = 0$. The probability that a firm producing good $1$ is functional is simply

\begin{equation}
  p_1(S_1) = 1 - (1 - \mu)^{S_1},
\end{equation}

hence, as before, the firm will pick at random $S_1$ suppliers, such that,

\begin{equation}
  S_1 \leq \frac{\log(\kappa / \pi) - \log \mu}{\log(1 - \mu)}.
\end{equation}

A firm producing good 2 infers that producers of good 1 will pick $S_1$, but the fact that these do so at random implies that, in expectation, their probabilities of not being functional are correlated. Hence, in contrast to the case presented before, if the firm producing good $2$ picks $S_2$ suppliers $\{j_1, j_2 \ldots j_{S_2}\}$

\begin{equation}
  \E \Big[ p_2(S_2) \ \vert \ S_1 \Big] = 1 - \E \Big[ \P\left( j_1, j_2 \ldots j_{S_2} \notin \F \right)  \ \vert \ S_1 \Big] \leq 1 - p_1(S_1)^{S_2}.
\end{equation}

The expected joint probability $\E \Big[ \P\left( j_1, j_2 \ldots j_{S_2} \notin \F \right)\ \vert \ S_1 \Big]$ depends on the expected overlap in the suppliers of $\{j_1, j_2 \ldots j_{S_2}\}$. Letting $x_{i}$ be the set of suppliers of firms $j_i$, we can write the probability that the firms $\{j_1, j_2 \ldots j_{S_2}\}$ have $c$ suppliers in common as

\begin{equation}
  \sigma_c(S_2) \coloneqq \P \Bigg( \left\lvert \bigcap^{S_2}_{i = 1} x_i \right\rvert = c \Bigg) = \binom{m_0}{c} \binom{m_0}{S_1}^{S_2} \ \sum^{S_1 - c}_{j = 0} (-1)^j \binom{m - c}{j} \binom{m - c - j}{S_2 - c - j}.
\end{equation}

Then 

\begin{equation}
  \E \Big[ \P\left( j_1, j_2 \ldots j_{S_2} \notin \F \right)\ \vert \ S_1 \Big] = \sum^{S_1}_{c = 0} (1 - \mu)^{(S_1 - c) S_2 + c} \ \sigma_c(S_2).
\end{equation}

Hence, $S_2$ will be the biggest $S_2$ that satisfies 

\begin{equation}
  \begin{split}
    \frac{\kappa}{\pi} &\geq \E \Big[ p_2(S_2 + 1) \ \vert \ S_1 \Big] - \E \Big[ p_2(S_2) \ \vert \ S_1 \Big] \\
    &\geq \sum^{S_1}_{c = 0} (1 - \mu)^{(S_1 - c) S_2 + c} \ \Bigg( \sigma_c(S_2) - (1 - \mu)^{S_1} \ \sigma_c(S_2 + 1) \Bigg)
  \end{split}
\end{equation}


\subsection{Recursive formulation of vertical economy problem}

Each layer is indexed by $i \in \{0, 1, 2, 3, \ldots\}$. Let $\F_i$ be the set of functional nodes in layer $i$, with size $\abs{f_i}$. A firm in layer $i$ picks $s_i$ suppliers. Let 

\begin{equation}
  \P\Big( \abs{\F_i} = f_i \Big) = g_i(f_i).
\end{equation}

Furthermore, let $p_i(s_i)$ be the probability that a firm in layer $i$ is functional. Assume now that a firm in $i+1$ observe the distribution $g_i$. By picking a number of suppliers $s_{i + 1}$ for a given $f_i$, its probability of being functional is

\begin{equation}
  p_{i+1}(s_{i+1}, f_i) = \binom{m_i - s_{i+1}}{m_i - f_i} \binom{m_i}{m_i - f_i}^{-1}
\end{equation}

Hence, it will pick the maximum $s_{i+1}$ that satisfies

\begin{equation}
  \sum^{m_i}_{f_i = 0} g_i(f_i) \ \binom{m_i}{m_i - f_i}^{-1} \ \left( \binom{m_i - s_{i+1} + 1}{m_i - f_i} - \binom{m_i - s_{i+1}}{m_i - f_i} \right) \leq \frac{\kappa}{\pi}.
\end{equation}

Then the distribution over failures in the next node is

\begin{equation}
  g_{i+1}(f_{i+1}) = \sum^{m_i}_{f_i = 0} f_{i+1}^{p_{i+1}} \ (1 - f_{i + 1})^{(1 - p_{i + 1})}.
\end{equation}

% --- Bibliography
\newpage
\nocite{*}
\printbibliography

\newpage
\appendix
\section[Concavity of social planner problem]{$W(S)$ is concave}

Let $W: \Re^n \to \Re$ be

\begin{equation}
  W(S) = \sum^n_{i = 0} m_i \Big( \pi_i \ p_i(S) - \kappa S_i \Big).
\end{equation}

I will prove that $W$ is concave. Note that it is sufficient to prove that $p_i$ is concave. This can be proven by induction. The base case is 

\begin{equation}
  p_2(S) = (1 - \mu_2) \ (1 - \mu_1)^{S_2}
\end{equation}

\section{Belmann formulation}

There are two states. $p \in [0, 1]$ is the functional probability of the supplier firms. And 

\begin{equation}
  x \coloneqq \begin{pmatrix}
    \mu \\ m \ \pi
  \end{pmatrix} \in [0, 1] \times \Re_{>0}
\end{equation}

is the state vector of idiosyncratic layer characteristics. The two states evolve as

\begin{equation}
  p' = f(s, p, x) = (1 - x_1) \ (1 - (1 - p)^s)
\end{equation}

and

\begin{equation}
  x' = g(x).
\end{equation}

Then the value function can be written as 

\begin{equation}
  V(p, x) = \sup_{s \in \Re} \Big\{ x_2 \ f(s, p, x) - \kappa s + V\Big(f(s, p, x), g(x)\Big) \Big\}
\end{equation}

First it is convenient to compute

\begin{equation}
  \begin{split}
    \frac{\partial f}{\partial s}(s, p, x) &= -s (1 - x_1) (1 - p)^{s-1} \text{ and } \\
    \frac{\partial f}{\partial p}(s, p, x) &= -(1 - x_1) (1 - p)^{s}  \ln(1 - p). \\
  \end{split}
\end{equation}

The first order condition implies that a maximizer $\bar{s}$ needs to satisfy

\begin{equation} \label{app:eq:optimality}
  \frac{\partial f}{\partial s}(\bar{s}, p, x) \left(x_2 + \frac{\partial V}{\partial s}\Big(f(\bar{s}, p, x), g(x)\Big) \right) = \kappa.
\end{equation}

Using $\bar{s}$ we can further compute

\begin{equation}
  \begin{split}
    V(p, x) &=  x_2 \ f(\bar{s}, p, x) - \kappa s + V\Big(f(\bar{s}, p, x), g(x)\Big) \\
    \frac{\partial V}{\partial p}(p, x) &= \left(x_2 + \frac{\partial V}{\partial s}\Big(f(\bar{s}, p, x), g(x)\Big) \right) \ \frac{\partial f}{\partial p}(s, p, x).
  \end{split}
\end{equation}

We can use this in equation (\ref{app:eq:optimality}) to obtain

\begin{equation}
  \frac{\partial V}{\partial p}(p, x) = \frac{\kappa \ln(1 - p) (1 - p)}{\bar{s}}
\end{equation}



\end{document}